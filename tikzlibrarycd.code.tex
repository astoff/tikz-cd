% tikzlibrarycd.code.tex [2018/11/19 v0.9f Commutative diagrams with TikZ]
%
% Copyright 2011, 2012, 2014, 2018 by Augusto Stoffel
%
% This program is free software: you can redistribute it and/or modify
% it under the terms of the GNU General Public License as published by
% the Free Software Foundation, either version 3 of the License, or
% (at your option) any later version.
%
% This program is distributed in the hope that it will be useful,
% but WITHOUT ANY WARRANTY; without even the implied warranty of
% MERCHANTABILITY or FITNESS FOR A PARTICULAR PURPOSE.  See the
% GNU General Public License for more details.
%
% You should have received a copy of the GNU General Public License
% along with this program.  If not, see <http://www.gnu.org/licenses/>.

\usetikzlibrary{matrix,quotes,arrows.meta}

\newif\iftikzcd@mathmode

\def\tikzcdset{\pgfqkeys{/tikz/commutative diagrams}}

\tikzcdset{
  name/.style={/tikz/every matrix/.append style={name=#1}},
  arrows/.code={\tikzcdset{every arrow/.append style={#1}}},
  labels/.code={\tikzcdset{every label/.append style={#1}}},
  cells/.code={\tikzcdset{every cell/.append style={#1}}},
  diagrams/.code={\tikzcdset{every diagram/.append style={#1}}},
  execute before arrows/.code={%
    \expandafter\def%
    \expandafter\tikzcd@before@paths@hook%
    \expandafter{\tikzcd@before@paths@hook#1}},
  to/.code={\tikzcd@setarrowend\tikzcd@ar@target{#1}},
  from/.code={\tikzcd@setarrowend\tikzcd@ar@start{#1}},
  shorten/.style={
    /tikz/shorten <=#1,
    /tikz/shorten >=#1},
  description/.style={
    /tikz/anchor=center,
    /tikz/fill=\pgfkeysvalueof{/tikz/commutative diagrams/background color}},
  marking/.style={
    /tikz/font=,
    /tikz/sloped,
    /tikz/shape=asymmetrical rectangle,
    /tikz/anchor=center},
  phantom/.style={
    /tikz/draw=none,
    /tikz/commutative diagrams/labels={
      /tikz/font=,
      /tikz/anchor=center}},
  crossing over/.style={
    /tikz/preaction={
      /tikz/draw=\pgfkeysvalueof{/tikz/commutative diagrams/background color},
      /tikz/arrows=-,
      /tikz/line width=\pgfkeysvalueof{/tikz/commutative diagrams/crossing over clearance}}},
  cramped/.code={\tikzcdset{
    every matrix/.append style={inner sep=+-0.3em},
    every cell/.append style={inner sep=+0.3em}}},
  row sep/.code={\tikzcd@sep{row}{#1}},
  column sep/.code={\tikzcd@sep{column}{#1}},
  sep/.code={\tikzcdset{row sep={#1},column sep={#1}}},
  math mode/.is if=tikzcd@mathmode,
  arrow style/.is choice}

\def\tikzcd@sep#1#2{%
  \pgfkeysifdefined{/tikz/commutative diagrams/#1 sep/#2}%
    {\pgfkeysgetvalue{/tikz/commutative diagrams/#1 sep/#2}\tikzcd@temp%
     \pgfkeysalso{/tikz/#1 sep/.expand once=\tikzcd@temp}}%
    {\pgfkeysalso{/tikz/#1 sep={#2}}}}

\def\tikzcd@setarrowend#1#2{%
  \pgfutil@ifundefined{pgf@sh@ns@\tikzcdmatrixname-#2}%
    {%
      \c@pgf@counta=0%
      \c@pgf@countb=0%
      \let\tikzcd@temp=\tikzcd@parse%
      \expandafter\tikzcd@temp#2\relax%
      \ifx\tikzcd@temp\pgfutil@empty% true if it was a direction parameter
        \advance\c@pgf@counta by\tikzcd@currentrow%
        \advance\c@pgf@countb by\tikzcd@currentcolumn%
        \edef#1{\tikzcdmatrixname-\the\c@pgf@counta-\the\c@pgf@countb}%
      \else%
        \def#1{#2}%
      \fi%
    }{%
      \def#1{\tikzcdmatrixname-#2}%
    }}

% The unknown key handler and direction argument parser
\tikzcdset{
  .unknown/.code={%
    \ifpgfkeysaddeddefaultpath%
      \c@pgf@counta=0%
      \c@pgf@countb=0%
      \let\tikzcd@temp=\tikzcd@parse%
      \expandafter\tikzcd@temp\pgfkeyscurrentname\relax%
      \ifx\tikzcd@temp\pgfutil@empty% true if it was a direction parameter
        \advance\c@pgf@counta by\tikzcd@currentrow%
        \advance\c@pgf@countb by\tikzcd@currentcolumn%
        \edef\tikzcd@ar@target{\tikzcdmatrixname-\the\c@pgf@counta-\the\c@pgf@countb}%
      \else%
        \pgfqkeys{/tikz}{\pgfkeyscurrentname={#1}}%
      \fi%
    \else%
      \def\pgfutilnext{\pgfkeysvalueof{/handlers/.unknown/.@cmd}#1\pgfeov}\pgfutilnext%
    \fi}}

\def\tikzcd@parse#1{% parser for arrow direction argument
  \ifx#1r\advance\c@pgf@countb by1\else%
   \ifx#1d\advance\c@pgf@counta by1\else%
    \ifx#1l\advance\c@pgf@countb by-1\else%
     \ifx#1u\advance\c@pgf@counta by-1\else%
      \ifx#1\relax\let\tikzcd@temp\pgfutil@empty%
       \else\let\tikzcd@temp\pgfutil@gobble@until@relax\fi\fi\fi\fi\fi%
  \tikzcd@temp}

% The `tikzcd' environment
\def\tikzcd{%
  \iffalse{\fi\ifnum0=`}\fi% standard trick to allow nesting of tabular-like environments (cf. the tabularx manual)
  \pgfutil@ifnextchar&% if we get an & here, it will be returned with the wrong catcode to the input stream
    {\expandafter\tikzcd@@\expandafter\pgfmatrixnextcell\pgfutil@gobble}%
    {\tikzcd@@}}

\def\tikzcd@@{\pgfutil@ifnextchar[{\tikzcd@handle@shortcuts@next\tikzcd@}{\tikzcd@[]}}

\def\tikzcd@[#1]{%
  \tikzifinpicture{%
    \let\tikzcdpicture\scope\let\endtikzcdpicture\endscope% <-- inside tikzpicture : start a scope
  }{%
    \let\tikzcdpicture\tikzpicture\let\endtikzcdpicture\endtikzpicture% <-- outside tikzpicture : start a tikzpicture
  }%
  \tikzcdpicture[/tikz/commutative diagrams/.cd,every diagram,#1]%
  \ifx\arrow\tikzcd@arrow%
    \pgfutil@packageerror{tikz-cd}{Diagrams cannot be nested}{}%
  \fi%
  \let\arrow\tikzcd@arrow%
  \let\ar\tikzcd@arrow%
  \def\rar{\tikzcd@xar{r}}%
  \def\lar{\tikzcd@xar{l}}%
  \def\dar{\tikzcd@xar{d}}%
  \def\uar{\tikzcd@xar{u}}%
  \def\urar{\tikzcd@xar{ur}}%
  \def\ular{\tikzcd@xar{ul}}%
  \def\drar{\tikzcd@xar{dr}}%
  \def\dlar{\tikzcd@xar{dl}}%
  \global\let\tikzcd@savedpaths\pgfutil@empty%
  \matrix[%
    /tikz/matrix of \iftikzcd@mathmode math \fi nodes,%
    /tikz/every cell/.append code={\tikzcdset{every cell}},%
    /tikz/commutative diagrams/.cd,every matrix]%
  \bgroup}

\def\endtikzcd{%
  \pgfmatrixendrow\egroup%
  \pgfextra{\global\let\tikzcdmatrixname\tikzlastnode};%
  \tikzcdset{\the\pgfmatrixcurrentrow-row diagram/.try}%
  \begingroup%
    \pgfkeys{% `quotes' library support
      /handlers/first char syntax/the character "/.initial=\tikzcd@forward@quotes,%
      /tikz/edge quotes mean={%
        edge node={node [execute at begin node=\iftikzcd@mathmode$\fi,%$
                         execute at end node=\iftikzcd@mathmode$\fi,%$
                         /tikz/commutative diagrams/.cd,every label,##2]{##1}}}}%
    \let\tikzcd@errmessage\errmessage% improve error messages
    \def\errmessage##1{\tikzcd@errmessage{##1^^J...^^Jl.\tikzcd@lineno\space%
        I think the culprit is a tikzcd arrow in cell \tikzcd@currentrow-\tikzcd@currentcolumn}}%
    \tikzcd@before@paths@hook%
    \tikzcd@savedpaths%
  \endgroup%
  \endtikzcdpicture%
  \ifnum0=`{}\fi}

% The arrow commands
\def\tikzcd@arrow{%
  \relax% this was added to avoid errors when a cell starts with \arrow, but it seems unnecessary now
  \edef\tikzcd@temp{%
    \noexpand\def\noexpand\tikzcd@currentcolumn{\the\pgfmatrixcurrentcolumn}%
    \noexpand\def\noexpand\tikzcd@currentrow{\the\pgfmatrixcurrentrow}%
    \noexpand\def\noexpand\tikzcd@lineno{\the\inputlineno}}%
  \expandafter\pgfutil@g@addto@macro\expandafter\tikzcd@savedpaths\expandafter{\tikzcd@temp}%
  \pgfutil@ifnextchar[{\tikzcd@handle@shortcuts@next\tikzcd@@arrow}{\tikzcd@ar@old[]}}

\def\tikzcd@@arrow[#1]{\pgfutil@ifnextchar\bgroup{\tikzcd@ar@old[#1]}{\tikzcd@ar@new[#1]}}

\def\tikzcd@ar@new[#1]{% new syntax
  \pgfutil@g@addto@macro\tikzcd@savedpaths{%
    \path[/tikz/commutative diagrams/.cd,every arrow,#1]%
    (\tikzcd@ar@start\tikzcd@startanchor) to (\tikzcd@ar@target\tikzcd@endanchor); }}

\def\tikzcd@ar@old[#1]#2{% old syntax
  \pgfutil@ifnextchar[%
    {\tikzcd@handle@shortcuts@next\tikzcd@ar@getlabel{to={#2},#1}}%
    {\pgfutil@ifnextchar\bgroup%
      {\tikzcd@ar@getlabel{to={#2},#1}[]}%
      {\tikzcd@ar@new[to={#2},#1]}}}

\def\tikzcd@ar@getlabel#1[#2]#3{%
  \pgfutil@ifnextchar[%
    {\tikzcd@handle@shortcuts@next\tikzcd@ar@getlabel{#1,"{#3}"{#2}}}%
    {\pgfutil@ifnextchar\bgroup%
      {\tikzcd@ar@getlabel{#1,"{#3}"{#2}}[]}%
      {\tikzcd@ar@new[#1,"{#3}"{#2}]}}}

\def\tikzcd@xar#1{\relax\pgfutil@ifnextchar[{\tikzcd@handle@shortcuts@next\tikzcd@@xar{#1}}{\tikzcd@arrow[]{#1}}}
\def\tikzcd@@xar#1[#2]{\tikzcd@arrow[#2]{#1}}

\def\tikzcd@handle@shortcuts@next{% expand next macro using standard catcodes
  \iftikz@handle@active@code%
    \begingroup%
      \tikz@switchoff@shorthands\expandafter%
    \endgroup\expandafter%
  \fi}

\def\tikzcd@ar@target{\tikzcdmatrixname-\tikzcd@currentrow-\tikzcd@currentcolumn}
\def\tikzcd@ar@start{\tikzcdmatrixname-\tikzcd@currentrow-\tikzcd@currentcolumn}

\def\tikzcd@forward@quotes#1{\tikzset{every to/.append style={#1}}}

\let\tikzcd@before@paths@hook\pgfutil@empty

% `start anchor', `end anchor', and `shift' keys
\def\tikzcd@setanchor#1[#2]#3\relax{%
  \ifx\relax#2\relax\else%
    \tikzcdset{@#1transform/.append style={#2},@shiftabletopath}%
  \fi%
  \ifx\relax#3\relax%
    \pgfutil@namelet{tikzcd@#1anchor}{pgfutil@empty}%
  \else%
    \pgfutil@namedef{tikzcd@#1anchor}{.#3}%
  \fi}

\tikzcdset{
  @shiftabletopath/.style={
    /tikz/execute at begin to={%
      \begingroup%
        \def\tikz@tonodes{coordinate[pos=0,commutative diagrams/@starttransform/.try](tikzcd@nodea) %
                          coordinate[pos=1,commutative diagrams/@endtransform/.try](tikzcd@nodeb)}%
        \path (\tikztostart) \tikz@to@path;%
      \endgroup%
      \def\tikztostart{tikzcd@nodea}%
      \def\tikztotarget{tikzcd@nodeb}%
      \tikzset{insert path={(tikzcd@nodea)}}},
    /tikz/commutative diagrams/@shiftabletopath/.code={}},
  start anchor/.code={%
    \pgfutil@ifnextchar[{\tikzcd@setanchor{start}}{\tikzcd@setanchor{start}[]}#1\relax},%]
  end anchor/.code={%
    \pgfutil@ifnextchar[{\tikzcd@setanchor{end}}{\tikzcd@setanchor{end}[]}#1\relax},%]
  start anchor=,
  end anchor=,
  shift left/.style={
    /tikz/commutative diagrams/@shiftabletopath,
    /tikz/execute at begin to={%
      \pgfpointnormalised{%
        \pgfpointdiff{\pgfpointanchor{tikzcd@nodeb}{center}}{\pgfpointanchor{tikzcd@nodea}{center}}}%
      \pgfgetlastxy{\tikzcd@x}{\tikzcd@y}%
      \pgfmathparse{#1}%
      \ifpgfmathunitsdeclared\else
        \pgfmathparse{\pgfmathresult*\pgfkeysvalueof{/tikz/commutative diagrams/shift left/.@def}}
      \fi
      \coordinate (tikzcd@nodea) at ([shift={(\pgfmathresult*\tikzcd@y,-\pgfmathresult*\tikzcd@x)}]tikzcd@nodea);%
      \coordinate (tikzcd@nodeb) at ([shift={(\pgfmathresult*\tikzcd@y,-\pgfmathresult*\tikzcd@x)}]tikzcd@nodeb);%
      \tikzset{insert path={(tikzcd@nodea)}}}},
  shift right/.style={
    /tikz/commutative diagrams/shift left={-(#1)}},
  transform nodes/.style={
    /tikz/commutative diagrams/@shiftabletopath,
    /tikz/commutative diagrams/@starttransform/.append style={#1},
    /tikz/commutative diagrams/@endtransform/.append style={#1}},
  shift/.style={
    /tikz/shift={#1},% we need this for labels to receive a shift
    /tikz/commutative diagrams/transform nodes={/tikz/shift={#1}}},
  xshift/.style={
    /tikz/xshift={#1},
    /tikz/commutative diagrams/transform nodes={/tikz/xshift={#1}}},
  yshift/.style={
    /tikz/yshift={#1},
    /tikz/commutative diagrams/transform nodes={/tikz/yshift={#1}}}}

% pgfmath functions to obtain math fontdimens
\pgfutil@ifluatex

  \directlua{tex.enableprimitives('tikzcd@', {'Umathaxis', 'Umathfractionrule'})}

  \pgfmathdeclarefunction{axis_height}{0}{%
    \begingroup%
      $\relax$% update fontdimens
      \pgfmathreturn\the\tikzcd@Umathaxis\textstyle%
    \endgroup}

  \pgfmathdeclarefunction{rule_thickness}{0}{%
    \begingroup%
      $\relax$%
      \pgfmathreturn\the\tikzcd@Umathfractionrule\textstyle%
    \endgroup}

\else% if using (pdf)tex

  \pgfmathdeclarefunction{axis_height}{0}{%
    \begingroup%
      $\relax$% update fontdimens
      \pgfmathreturn\the\fontdimen22\textfont2%
    \endgroup}

  \pgfmathdeclarefunction{rule_thickness}{0}{%
    \begingroup%
      $\relax$%
      \pgfmathreturn\the\fontdimen8\textfont3%
    \endgroup}

\fi

% The shape used inside matrices
\pgfdeclareshape{asymmetrical rectangle} {
  \inheritsavedanchors[from={rectangle}]
  \inheritanchor[from={rectangle}]{base}
  \inheritanchor[from={rectangle}]{north}
  \inheritanchor[from={rectangle}]{south}
  \inheritanchor[from={rectangle}]{base west}
  \inheritanchor[from={rectangle}]{north west}
  \inheritanchor[from={rectangle}]{south west}
  \inheritanchor[from={rectangle}]{base east}
  \inheritanchor[from={rectangle}]{north east}
  \inheritanchor[from={rectangle}]{south east}
  \inheritanchor[from={rectangle}]{mid}
  \inheritanchor[from={rectangle}]{mid west}
  \inheritanchor[from={rectangle}]{mid east}
  \inheritbackgroundpath[from={rectangle}]
  \anchor{center}{\pgf@anchor@rectangle@center\pgfmathsetlength\pgf@y%
    {\pgfkeysvalueof{/tikz/commutative diagrams/center yshift}}}
  \anchor{west}{\pgf@anchor@rectangle@west\pgfmathsetlength\pgf@y%
    {\pgfkeysvalueof{/tikz/commutative diagrams/center yshift}}}
  \anchor{east}{\pgf@anchor@rectangle@east\pgfmathsetlength\pgf@y%
    {\pgfkeysvalueof{/tikz/commutative diagrams/center yshift}}}
  \anchor{real center}{\pgf@anchor@rectangle@center}
  \anchor{real west}{\pgf@anchor@rectangle@west}
  \anchor{real east}{\pgf@anchor@rectangle@east}
  \anchorborder{%
    \pgfmathsetlength\pgfutil@tempdima%
      {\pgfkeysvalueof{/tikz/commutative diagrams/center yshift}}%
    \pgf@xb=\pgf@x% xb/yb is target
    \pgf@yb=\pgf@y%
    \southwest%
    \pgf@xa=\pgf@x% xa/ya is se
    \pgf@ya=\pgf@y%
    \northeast%
    \advance\pgf@x by-\pgf@xa%
    \advance\pgf@y by-\pgf@ya%
    \pgf@xc=.5\pgf@x% x/y is half width/height
    \pgf@yc=.5\pgf@y%
    \advance\pgf@xa by\pgf@xc% xa/ya becomes center
    \advance\pgf@ya by\pgf@yc%
    \ifdim\pgf@yb>0pt%
      \northeast%
      \pgf@yc=\pgf@y%
      \advance\pgf@yc by-\pgfutil@tempdima%
    \else%
      \southwest%
      \pgf@yc=-\pgf@y%
      \advance\pgf@yc by\pgfutil@tempdima%
    \fi
    \edef\pgf@marshal{%
      \noexpand\pgfpointborderrectangle
      {\noexpand\pgfqpoint{\the\pgf@xb}{\the\pgf@yb}}
      {\noexpand\pgfqpoint{\the\pgf@xc}{\the\pgf@yc}}%
    }%
    \pgf@process{\pgf@marshal}%
    \advance\pgf@x by\pgf@xa%
    \advance\pgf@y by\pgfutil@tempdima%
  }%
}%

% Computer modern arrow tips
\pgfkeys{
  cm to/.tip={Computer Modern Rightarrow[length=+0pt 6.2]},
  cm bold to/.tip={cm to[scale=0.667]},
  cm double to/.tip={cm to[sep=+0pt -2.6]cm to},
  cm bar/.tip={Bar[width=+0pt 8.2 0.89,line cap=round]},
  cm left hook/.tip={Hooks[width=+0pt 10.8,length=+0pt 3.6,harpoon,line cap=round]},
  cm implies/.tip={Implies},
  cm left to/.tip={cm to[left]},
  cm right to/.tip={cm to[right]},
  cm right hook/.tip={cm left hook[swap]},
  cm to reversed/.tip={cm to[reversed]},
  cm */.tip={Circle[length=+0pt 7.6]},
  cm o/.tip={cm *[open]},
  cm |/.tip={cm bar}}%|

% The `Glyph' meta arrow tip
\pgfqkeys{/pgf/arrow keys}{
  glyph math command/.code={% if csname #1 is undefined, we want it to remain so
    \pgfarrowsaddtooptions{\def\tikzcd@glyph{$\begingroup\expandafter\endgroup\csname #1\endcsname$}}},
  glyph axis/.code={\pgfarrowsaddtooptions{\pgfmathsetlengthmacro\tikzcd@glyph@axis{#1}}},
  glyph length/.code={\pgfarrowsaddtooptions{\pgfmathsetlengthmacro\tikzcd@glyph@len{#1}}},
  glyph shorten/.code={\pgfarrowsaddtooptions{\pgfmathsetlengthmacro\tikzcd@glyph@shorten{#1}}}}

\pgfdeclarearrow{
  name=Glyph,
  cache=false,
  bending mode=none,
  parameters={\tikzcd@glyph@len,\tikzcd@glyph@shorten},
  setup code={%
    \pgfarrowssettipend{\tikzcd@glyph@len\advance\pgf@x by\tikzcd@glyph@shorten}},
  defaults={
    glyph axis=axis_height,
    glyph length=+0.9ex,
    glyph shorten=+-0.1ex},
  drawing code={%
    \pgfpathrectangle{\pgfpoint{+0pt}{+-1ex}}{\pgfpoint{+\tikzcd@glyph@len}{+2ex}}%
    \pgfusepathqclip%
    \pgftransformxshift{+\tikzcd@glyph@len}%
    \pgftransformyshift{+-\tikzcd@glyph@axis}%
    \pgftext[right,base]{\tikzcd@glyph}}}

% Arrow tip selection styles
\tikzcdset{
  double line/.code={\tikzset{%
    double equal sign distance,
    double=\pgfkeysvalueof{/tikz/commutative diagrams/background color}}},
  dashed/.style={/tikz/dash pattern={on 7\pgflinewidth off 4\pgflinewidth}},
  tikzcd to/.tip={cm to},
  tikzcd bar/.tip={cm bar},
  tikzcd left hook/.tip={cm left hook},
  tikzcd double to/.tip={cm double to},
  tikzcd implies/.tip={Implies},
  tikzcd right hook/.tip={tikzcd left hook[swap]},
  tikzcd left to/.tip={tikzcd to[harpoon]},
  tikzcd right to/.tip={tikzcd left to[swap]},
  tikzcd to reversed/.tip={tikzcd to[reversed]},
  tikzcd cap/.tip={},
  tikzcd implies cap/.tip={},
  tikzcd implies bar/.tip={tikzcd bar},
  no tail/.code={\pgfsetarrowsstart{tikzcd cap}},
  to head/.code={\pgfsetarrowsend{tikzcd to}},
  maps to/.code={\pgfsetarrowsstart{tikzcd bar}},
  hook/.code={\pgfsetarrowsstart{tikzcd right hook}},
  hook'/.code={\pgfsetarrowsstart{tikzcd left hook}},
  harpoon/.code={\pgfsetarrowsend{tikzcd left to}},
  harpoon'/.code={\pgfsetarrowsend{tikzcd right to}},
  two heads/.code={\pgfsetarrowsend{tikzcd double to}},
  tail/.code={\pgfsetarrowsstart{tikzcd to reversed}},
  rightarrow/.code={\pgfsetarrows{tikzcd cap-tikzcd to}},
  Rightarrow/.code={\tikzcdset{double line}\pgfsetarrows{tikzcd implies cap-tikzcd implies}},
  leftarrow/.code={\pgfsetarrows{tikzcd to-tikzcd cap}},
  Leftarrow/.code={\tikzcdset{double line}\pgfsetarrows{tikzcd implies-tikzcd implies cap}},
  leftrightarrow/.code={\pgfsetarrows{tikzcd to-tikzcd to}},
  Leftrightarrow/.code={\tikzcdset{double line}\pgfsetarrows{tikzcd implies-tikzcd implies}},
  mapsto/.code={\pgfsetarrows{tikzcd bar-tikzcd to}},
  mapsfrom/.code={\pgfsetarrows{tikzcd to-tikzcd bar}},
  Mapsto/.code={\tikzcdset{double line}\pgfsetarrows{tikzcd implies bar-tikzcd implies}},
  Mapsfrom/.code={\tikzcdset{double line}\pgfsetarrows{tikzcd implies-tikzcd implies bar}},
  hookrightarrow/.code={\pgfsetarrows{tikzcd right hook-tikzcd to}},
  hookleftarrow/.code={\pgfsetarrows{tikzcd to-tikzcd left hook}},
  rightharpoonup/.code={\pgfsetarrows{tikzcd cap-tikzcd left to}},
  rightharpoondown/.code={\pgfsetarrows{tikzcd cap-tikzcd right to}},
  leftharpoonup/.code={\pgfsetarrows{tikzcd right to-tikzcd cap}},
  leftharpoondown/.code={\pgfsetarrows{tikzcd left to-tikzcd cap}},
  rightarrowtail/.code={\pgfsetarrows{tikzcd to reversed-tikzcd to}},
  leftarrowtail/.code={\pgfsetarrows{tikzcd to-tikzcd to reversed}},
  twoheadrightarrow/.code={\pgfsetarrows{tikzcd cap-tikzcd double to}},
  twoheadleftarrow/.code={\pgfsetarrows{tikzcd double to-tikzcd cap}},
  no head/.code={\pgfsetarrowsend{tikzcd cap}},
  dash/.code={\pgfsetarrows{tikzcd cap-tikzcd cap}},
  dashrightarrow/.code={\tikzcdset{rightarrow,dashed}},
  dashleftarrow/.code={\tikzcdset{leftarrow,dashed}},
  equal/.code={\tikzcdset{double line}\pgfsetarrows{tikzcd implies cap-tikzcd implies cap}},
  equals/.code={\tikzcdset{equal}},
  rightsquigarrow/.code={\tikzcdset{rightarrow,squiggly}},
  leftsquigarrow/.code={\tikzcdset{leftarrow,squiggly}},
  leftrightsquigarrow/.code={\tikzcdset{leftrightarrow,squiggly}},
  squiggly/.code={%
    \pgfutil@ifundefined{tikz@library@decorations.pathmorphing@loaded}%
      {\pgfutil@packageerror{tikz-cd}{You need to say %
      \string\usetikzlibrary{decorations.pathmorphing} to use squiggly arrows}{}}{}%
    \pgfkeysalso{
      /tikz/decorate,
      /tikz/decoration={
        zigzag,
        segment length=9.25\pgflinewidth,
        amplitude=1.9\pgflinewidth,
        post=lineto, post length=6\pgflinewidth,
        pre=lineto, pre length=6\pgflinewidth,
        #1}}}}

% The `Latin Modern' arrow style
\pgfkeysdef{/tikz/commutative diagrams/arrow style/Latin Modern}{%
  \tikzcdset{
    tikzcd bar/.tip={Bar[width=+0pt 12 .992,line cap=round]},
    tikzcd left hook/.tip={Hooks[width=+0pt 15,length=+0pt 4.2,arc=190,harpoon,line cap=round]}}}

% The `tikz' arrow style
\pgfkeysdef{/tikz/commutative diagrams/arrow style/tikz}{%
  \tikzcdset{
    tikzcd to/.tip={>},
    tikzcd bar/.tip={Bar[width=+3pt 4 0.9]},
    tikzcd left hook/.tip={Hooks[harpoon]},
    tikzcd double to/.tip={tikzcd to[]tikzcd to}}}

% The `math font' arrow style
%
% If your font does not define the command \mapsfrom, do so by saying
%   \def\mapsfrom{\pgfpicture\pgfsetbaseline{+0pt}\pgftransformxscale{-1}\pgftext[base]{$\mapsto$}\endpgfpicture}
%
% The method used to guess the correct value of `double distance'
% inside the `double line' style is not robust at all.  If it fails,
% set
%   \tikzcdset{double line/.append style={double distance=2*rule_thickness}}
% and find the correct value for `double distance' by trial and error
% (or measure the distance between the \Rightarrow stems in a font
% editor).
%
\pgfkeysdef{/tikz/commutative diagrams/arrow style/math font}{%
  \tikzcdset{
    tikzcd to/.tip={Glyph[glyph math command=rightarrow]},
    tikzcd cap/.tip={Glyph[glyph math command=leftarrow]},
    tikzcd to reversed/.tip={Glyph[glyph math command=leftarrowtail]},
    tikzcd bar/.tip={Glyph[glyph math command=mapsfrom]},
    tikzcd left hook/.tip={Glyph[glyph math command=hookleftarrow]},
    tikzcd right hook/.tip={Glyph[glyph math command=hookleftarrow, swap]},
    tikzcd implies/.tip={Glyph[glyph math command=Rightarrow, glyph length=1.2ex]},
    tikzcd implies cap/.tip={Glyph[glyph math command=Leftarrow]},
    tikzcd implies bar/.tip={Glyph[glyph math command=Mapsfrom]},
    tikzcd double to/.tip={Glyph[glyph math command=twoheadrightarrow, glyph length=1.4ex]},
    tikzcd left to/.tip={Glyph[glyph math command=rightharpoonup]},
    tikzcd right to/.tip={Glyph[glyph math command=rightharpoonup,swap]},
    double line/.append code={\tikzset{double distance={2*(height("$=$")-axis_height-rule_thickness)}}},
    dashed/.code={\tikzset{dash pattern=on 0.8ex off 0.4ex, dash phase=0.8ex}},
    squiggly/.default={pre length=1ex, post length=1ex}}}

% Default settings
\tikzcdset{
  every arrow/.style={
    /tikz/draw,
    /tikz/line width=rule_thickness,
    /tikz/commutative diagrams/rightarrow},
  every label/.style={
    /tikz/auto,
    /tikz/font=\everymath\expandafter{\the\everymath\scriptstyle},
    /tikz/inner sep=+0.5ex},
  every cell/.style={
    /tikz/shape={asymmetrical rectangle},
    /tikz/inner xsep=+1ex,
    /tikz/inner ysep=+0.85ex},
  every matrix/.style={/tikz/inner sep=+0pt},
  every diagram/.style={
    /tikz/commutative diagrams/row sep=normal,
    /tikz/commutative diagrams/column sep=normal,
    /tikz/baseline=+0pt},
  1-row diagram/.style={/tikz/baseline/.expanded=(\tikzcdmatrixname.base)},
  math mode=true,
  center yshift/.initial=axis_height,
  row sep/huge/.initial=+3.6em,
  row sep/large/.initial=+2.7em,
  row sep/normal/.initial=+1.8em,
  row sep/scriptsize/.initial=+1.35em,
  row sep/small/.initial=+0.9em,
  row sep/tiny/.initial=+0.45em,
  column sep/huge/.initial=+4.8em,
  column sep/large/.initial=+3.6em,
  column sep/normal/.initial=+2.4em,
  column sep/scriptsize/.initial=+1.8em,
  column sep/small/.initial=+1.2em,
  column sep/tiny/.initial=+0.6em,
  crossing over clearance/.initial=+1.5ex,
  shift left/.default=+0.56ex,
  shift right/.default=1,
  background color/.initial=white}

% ConTeXt-specific stuff
\pgfutil@IfUndefined{starttikzpicture}{}{%
  \def\starttikzcd{\tikzcd}
  \def\stoptikzcd{\endtikzcd}
  \tikzcdset{
    every matrix/.append code={%
      \def\NC{\pgfmatrixnextcell}%
      \def\NR{\pgfmatrixendrow}}}
}

\endinput
